\documentclass{peterlitsdoc}


%%%%%%%%%%%%%%%%%%%%%%%%%%%%%%%%%%%%%%%%%%%%%%%%%%%%%%%%%%%%%%%%%%%%%%%%%%%%%%%

\usepackage{datenumber}

\makeatletter

\newcounter{plt@date@temp}
\newcounter{plt@date@today}
\setmydatenumber{plt@date@today}{\the\year}{\the\month}{\the\day}
\newcounter{plt@date@end}
\newlength{\plt@date@unit}
\newlength{\plt@date@tempLength}

\newcommand{\plt@date@daydiff}[3]{%
    \setmydatenumber{plt@date@temp}{#1}{#2}{#3}%
    \addtocounter{plt@date@temp}{-\theplt@date@today}%
}

\newcommand{\plt@date@setend}[3]{%
    \setmydatenumber{plt@date@end}{#1}{#2}{#3}%
    \plt@date@daydiff{#1}{#2}{#3}%
    % assert that the end is bigger than today
    \setlength{\plt@date@unit}{0.9\textwidth/\theplt@date@temp}
}

\newcommand{\pltdate}[5][below]{%
    \plt@date@daydiff{#2}{#3}{#4}%
    \ifnum\theplt@date@temp > -1%
        \setlength{\plt@date@tempLength}{\plt@date@unit*\theplt@date@temp}%
        \fill(\plt@date@tempLength,0) circle [radius=1pt];%
        \node [#1] at (\the\plt@date@tempLength,0) {\parbox{%
            \widthof{\footnotesize#2-#3-#4}%
        }{%
            \centering\footnotesize#2-#3-#4\par%
            \fbox{\parbox[t]{\widthof{\footnotesize#2-#3-#4}-2\fboxsep}%
                {\scriptsize#5}%
            }%
        }}
    \fi%
}

\newenvironment{plttimeline}[3]{%
    \def\D{\pltdate}
    \plt@date@setend{#1}{#2}{#3}%
    \begin{center}
    \begin{tikzpicture}
    \draw (0,0) -- (0.9\textwidth,0);
    \pltdate[above]{\the\year}{\the\month}{\the\day}{NOW};
    \pltdate[above]{#1}{#2}{#3}{END};
}{%
    \end{tikzpicture}
    \end{center}
    \def\D{\undefined}
}

\makeatother

%%%%%%%%%%%%%%%%%%%%%%%%%%%%%%%%%%%%%%%%%%%%%%%%%%%%%%%%%%%%%%%%%%%%%%%%%%%%%%%

\title{时间记录}
\author{Peterlits Zo}

\begin{document}

\maketitle
\tableofcontents
\newpage

%%%%%%%%%%%%%%%%%%%%%%%%%%%%%%%%%%%%%%%%%%%%%%%%%%%%%%%%%%%%%%%%%%%%%%%%%%%%%%%
% MAIN PART %%%%%%%%%%%%%%%%%%%%%%%%%%%%%%%%%%%%%%%%%%%%%%%%%%%%%%%%%%%%%%%%%%%
%%%%%%%%%%%%%%%%%%%%%%%%%%%%%%%%%%%%%%%%%%%%%%%%%%%%%%%%%%%%%%%%%%%%%%%%%%%%%%%

\section{To-Do列表}

\subsection{时间线表}

\begin{plttimeline}{2020}{7}{8}
    \D{2020}{6}{3}{计算机考试};
    \D{2020}{6}{7}{英语作业};
    \D{2020}{6}{28}{中国近现代史纲要考试};
    \D{2020}{7}{6}{高考};
\end{plttimeline}

\makeatletter
\newcounter{Pmonth}
\newcounter{Pday}
\newcounter{Pyear}
\newcounter{Pend}
\setcounter{Pend}{\theplt@date@today+20}
\setmydatebynumber{\thePend}{Pyear}{Pmonth}{Pday}

\newcommand{\PD}[6][below]{
    \setdate{#2}{#3}{#4}
    \setcounter{datenumber}{\thedatenumber+1}
    \setdatebynumber{\thedatenumber}
    \D[#1]{\thedateyear}{\thedatemonth}{\thedateday}{#5};
    \setcounter{datenumber}{\thedatenumber+1}
    \setdatebynumber{\thedatenumber}
    \D[below=#6, #1]{\thedateyear}{\thedatemonth}{\thedateday}{#5};
    \setcounter{datenumber}{\thedatenumber+2}
    \setdatebynumber{\thedatenumber}
    \D[#1]{\thedateyear}{\thedatemonth}{\thedateday}{#5};
    \setcounter{datenumber}{\thedatenumber+6}
    \setdatebynumber{\thedatenumber}
    \D[#1]{\thedateyear}{\thedatemonth}{\thedateday}{#5};
}
\makeatother

\begin{comment}

\subsection{2-3-5-11表}

\begin{plttimeline}{\thePyear}{\thePmonth}{\thePday}
    \PD{2020}{6}{2}{测试}{3em};
\end{plttimeline}

\end{comment}

\subsection{To-do表}

\begin{plttodoenv}{4}
%                   %                   %                   %
\t[v]英语作业       \t[ ]数学作业       \t[ ]选体育课       \t[x]中医学视频
\t[ ]水MIT视频      \t[v]历史论文       \t[ ]历史视频       \t[ ]ACM改错
\t[ ]MIT作业        \t[ ]青年大学习
\end{plttodoenv}



%%%%%%%%%%%%%%%%%%%%%%%%%%%%%%%%%%%%%%%%%%%%%%%%%%%%%%%%%%%%%%%%%%%%%%%%%%%%%%%

\section{Today}

\subsection{时间表}
\begin{pltplan}
\item[x]{14:30}{数学作业}{开始做作业!我要加油!写\LaTeX{}写了很久很久
    我不应该这样子的!我还有重要的事情不能一直玩这个。哎。}
\item[x]{2:40}{中医学视频}{凌晨起来打卡。。。}
\item[ ]{}{数学作业}{}
\item[ ]{}{中医学视频}{}
\item[ ]{}{数学作业}{}
\item[ ]{}{中医学视频}{}
\item[ ]{}{水MIT视频}{}
\item[ ]{}{跑步}{}
\item[ ]{}{ACM改错}{}
\end{pltplan}

\subsection{正文}

好好上课!好好补上之前的东西。

搞了一个奇怪的时间表。我好傻哦。

\section{2020-5-31}

\subsection{时间表}

\begin{pltplan}
\item[x]{19:19}{英语作业}{做了一半没有电了。回来继续做。}
\item[v]{20:35}{英语作业}{我做完了!虽然现在还没有交给老师。}
\item[v]{21:00}{写小说}{本来应该直接去跑步的,耽误了一哈哈。}
\item[v]{22:10}{跑步}{估计跑完了就10点了吧。}
\item[v]{XX:XX}{英语作业}{提前做完了(见20:35)。}
\item[v]{00:00}{历史论文}{一直写到了今天早上。}
\item[ ]{}{数学作业}{}
\item[ ]{}{水MIT视频}{}
\end{pltplan}

\subsection{正文}

活得将就。

%%%%%%%%%%%%%%%%%%%%%%%%%%%%%%%%%%%%%%%%%%%%%%%%%%%%%%%%%%%%%%%%%%%%%%%%%%%%%%%

\section{5-31}

\subsection{时间表}

\begin{pltplan}
\item[x]{15:43}{英语作业}{做了一半,突然电脑没有电了,看了看
    \verb|learning scala|,然后回来跑步了。}
\item[v]{19:22}{跑步}{呼呼$\sim$}
\end{pltplan}

\subsection{正文}

又在搞\LaTeX{},我简直是不务正业。

\end{document}
