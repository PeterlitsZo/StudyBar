\documentclass{peterlitsdoc}

\title{时间记录}
\author{Peterlits Zo}

\begin{document}

\maketitle
\tableofcontents
\newpage

%%%%%%%%%%%%%%%%%%%%%%%%%%%%%%%%%%%%%%%%%%%%%%%%%%%%%%%%%%%%%%%%%%%%
% MAIN PART %%%%%%%%%%%%%%%%%%%%%%%%%%%%%%%%%%%%%%%%%%%%%%%%%%%%%%%%
%%%%%%%%%%%%%%%%%%%%%%%%%%%%%%%%%%%%%%%%%%%%%%%%%%%%%%%%%%%%%%%%%%%%

\section{To-Do列表}

\subsection{时间线表}

\begin{plttimeline}{2020}{7}{8}
    \D{2020}{6}{3}{计算机考试};
    \D{2020}{6}{7}{英语作业};
    \D{2020}{6}{28}{中国近现代史纲要考试};
    \D{2020}{7}{6}{高考};
    \D{2020}{6}{24}{英语考试};
    \D[below=3em]{2020}{6}{7}{ACM排位赛};
    \D{2020}{6}{10}{物理作业};
    \D[below=4em]{2020}{6}{28}{概率统计};
\end{plttimeline}

\subsection{2-3-5-11表}

\subsubsection{高等数学}

\pltstudyline{2020}{6}{4}{无穷级数}

\pltstudyline{2020}{6}{3}{微积分}

\subsection{To-do表}

\begin{plttodoenv}{4}
%                   %                   %                   %
\t[ ]数学作业      
\t[ ]选体育课      
\t[x]中医学视频
\t[ ]水MIT视频                         
\t[ ]历史视频
\t[ ]MIT作业       
\t[ ]青年大学习                        
\t[ ]上法语课
\t[ ]信息检索作业  
\t[ ]去邮局        
\t[ ]宏包“正规”选项
\t[ ]宏包制表符对不齐
    \t[ ]note - 字符串的宏定义
    \t[ ]note - substr
    \t[ ]note - 小数位数
\end{plttodoenv}

关于宏包制表符可以参考:\url{https://www.tug.org/TUGboat/tb28-1/tb88flynn.pdf}
这三个note可以参考:\url{https://vjudge.net/contest/377090}


%%%%%%%%%%%%%%%%%%%%%%%%%%%%%%%%%%%%%%%%%%%%%%%%%%%%%%%%%%%%%%%%%%%%%

\section{Today}

\subsection{To-Do列表}

\begin{plttodoenv}{4}
    \t[x]数学课-正向级数    \t[x]中医学作业-绪论    \t[ ]数学作业
    \t[ ]打比赛     \t[ ]青年大学习     \t[ ]数学课-无穷级数
    \t[ ]跑步   \t[ ]背英语单词
\end{plttodoenv}

\subsection{时间表}

\begin{pltplan}
\item[v]{14:15}{打排位赛}{打完了,有好多错题需要订正。}
\end{pltplan}

\newpage

%%%%%%%%%%%%%%%%%%%%%%%%%%%%%%%%%%%%%%%%%%%%%%%%%%%%%%%%%%%%%%%%%%%%%

\section{Yesterday}

\subsection{To-Do列表}

\begin{plttodoenv}{4}
    \t[x]上数学课   \t[x]中医学作业     \t[v]英语作业   \t[ ]数学作业
    \t[ ]青年大学习     \t[ ]微积分,B   \t[ ]无穷级数   \t[v]去邮局
    \t[v]跑步   \t[v]背英语单词
\end{plttodoenv}

\subsection{时间表}

\begin{pltplan}
    \othe[v]{12:00}{不眠不休地搞\LaTeX{}}{我试着搞了一个那啥,
        一个自动化的模板机器,可以方便地生成LaTeX{}模板}
    \item[x]{15:25}{中医学作业}{学了绪论的1/4}
    \othe[ ]{15:55}{在网上愚蠢的找Pomorodo CLI软件}{不应该的呀}
    \item[x]{16:26}{英语作业}{做完了快速阅读(虽然我做得一点也不快}
    \item[v]{17:24}{英语作业}{做完了!我好想要一个Surface go啊,
        现在我的电脑太重了啦。}
    \othe[ ]{18:34}{回家}{早点回去也不知道干什么}
    \item[x]{19:47}{中医学作业}{学到了第24页。还有20多页。好多哦。}
    \item[v]{20:21}{跑步}{出去跑步!}
    \othe[ ]{22:00}{洗澡、吃晚饭、看闲书、睡觉}{}
    \othe[ ]{08:58}{起床、刷牙洗脸、去图书馆}{虽然出去得蛮早,但是
        还是有不少事耽搁了。总之信已经寄出去了。}
    \item[x]{11:58}{上数学课}{开始学到了比较审敛法的极限形式。}
    \item[x]{12:54}{中医学作业}{有点觉得Surface Pro会比较好一点,但是好贵
        啊。看到了第37页,还有11页,今天应该能够看得完!}
    \item[x]{13:31}{上数学课}{应该能上完正向级数吧?嘿嘿。并没有看完}
\end{pltplan}

\end{document}
