\documentclass{peterlitsdoc}

\title{时间记录}
\author{Peterlits Zo}

\newcommand{\pltbar}[3]{{%
    \setlength{\baselineskip}{0pt}%
    \par\noindent%
    \vspace*{0.5ex}%
    \par%
    \parbox[c]{0.15\textwidth}{#1}%
    \hfill%
    \parbox[c]{0.7\textwidth}{
        \begin{tikzpicture}
            \draw (0, 0) rectangle (0.6\textwidth, 1.5ex);
            \fill (0, 0) rectangle (#2 / #3 * 0.6\textwidth, 1.5ex);
            \node[left] (A) at (0.7\textwidth, 0.75ex) {%
                {\ttfamily%
                 \luaexec{
                     tex.print(string.format("\%.1f", #2 / #3 * 100));
                 }\%}
            };
        \end{tikzpicture}
    }%
    \par%
}}

\begin{document}

\maketitle
\tableofcontents
\newpage

%%%%%%%%%%%%%%%%%%%%%%%%%%%%%%%%%%%%%%%%%%%%%%%%%%%%%%%%%%%%%%%%%%%%
% MAIN PART %%%%%%%%%%%%%%%%%%%%%%%%%%%%%%%%%%%%%%%%%%%%%%%%%%%%%%%%
%%%%%%%%%%%%%%%%%%%%%%%%%%%%%%%%%%%%%%%%%%%%%%%%%%%%%%%%%%%%%%%%%%%%

\section{To-Do列表}

\subsection{时间线表}

\begin{plttimeline}{2020}{7}{8}
    \D{2020}{6}{3}{计算机考试};
    \D{2020}{6}{7}{英语作业};
    \D{2020}{6}{28}{中国近现代史纲要考试};
    \D{2020}{7}{6}{高考};
    \D{2020}{6}{24}{英语考试};
    \D[below=3em]{2020}{6}{7}{ACM排位赛};
    \D{2020}{6}{10}{物理作业};
    \D[below=4em]{2020}{6}{28}{概率统计考试};
    \D[below=2em]{2020}{6}{22}{数学考试};
\end{plttimeline}

\subsection{2-3-5-11表}

% \subsubsection{高等数学}
% 
% \pltstudyline{2020}{6}{4}{无穷级数}
% 
% \pltstudyline{2020}{6}{3}{微积分}

\subsection{To-do表}

\subsubsection{所有目标}

\bigskip

\begin{plttodoenv}{4}
%                   %                   %                   %
\t[x]数学作业      
\t[x]中医学视频
\t[ ]水MIT视频                         
\t[ ]历史视频
\t[ ]MIT作业       
\t[ ]青年大学习                        
\t[ ]上法语课
\t[ ]信息检索作业  
\t[ ]宏包“正规”选项
\t[ ]宏包制表符对不齐
\t[ ]note - 字符串的宏定义
\t[ ]note - substr
\t[ ]note - 小数位数
\t[ ]物理作业
\t[ ]Defer语言
\t[ ]soda管理器
\t[ ]纲要作业
\t[ ]形式与政策作业
\end{plttodoenv}

\subsubsection{长期目标}

关于一些长期目标:
\pltbar{数学作业}{1}{45} % 每一个单位都是一页
\pltbar{纲要作业}{0}{100} % 看不到目前
\pltbar{中医作业}{3}{28} % 每一个单位都是一课

\subsubsection{备注}

关于宏包制表符可以参考:\url{https://www.tug.org/TUGboat/tb28-1/tb88flynn.pdf}

这三个note可以参考:\url{https://vjudge.net/contest/377090}

\subsection{实体名单}

我发现我经常有忘带东西!我如果去图书馆的话,应该要带:

\begin{plttodoenv}{4}
\t[ ]笔记本(带电的)
\t[ ]笔
\t[ ]笔记本(不带电的)
\t[ ]手机
\t[ ]耳机和耳机盒
\t[ ]笔记本的线
\t[ ]手机的线
\t[ ]耳机的线
\end{plttodoenv}

\newpage

%%%%%%%%%%%%%%%%%%%%%%%%%%%%%%%%%%%%%%%%%%%%%%%%%%%%%%%%%%%%%%%%%%%%%

\section{Today}

\subsection{To-Do列表}

\begin{plttodoenv}{4}
    \t[ ]物理作业
    \t[ ]物理课 - ?
    \t[ ]数学课 - 无穷级数
    \t[ ]跑步
    \t[ ]背英语单词
    \t[ ]中医学作业 - 第四节
    \t[ ]历史作业 - 第一节
    \t[ ]数学作业 - 常数项级数
    \t[ ]数学课 - 常数项级数
    \t[ ]青年大学习
\end{plttodoenv}

\subsection{时间表}

\begin{pltplan}
\item[ ]{17:06}{物理课}{因为进度条的事被夸了!我很高兴,但是我
    还有好多作业呀。快学习吧$\sim$\\看了一节,还有三节,看完学习
    做作业$\sim$}
\othe[ ]{17:44}{回家}{好饿呀,回家锻炼!}
\item[ ]{19:33}{锻炼}{回来玩了比较久,喝了点低度酒,感觉不太好,
    酒这个东西,应该还是要在吃饭后再喝。
\end{pltplan}

\newpage

%%%%%%%%%%%%%%%%%%%%%%%%%%%%%%%%%%%%%%%%%%%%%%%%%%%%%%%%%%%%%%%%%%%%%

\section{Yesterday}

\subsection{To-Do列表}

\begin{plttodoenv}{4}
    \t[x]数学作业 - 正向级数
    \t[v]数学课 - 正向级数
    \t[ ]物理作业
    \t[ ]物理课 - ?
    \t[ ]数学课 - 无穷级数
    \t[v]跑步
    \t[v]背英语单词
    \t[v]中医学作业 - 第3节
    \t[v]选体育课
\end{plttodoenv}

\subsection{时间表}

\begin{pltplan}
\othe[ ]{15:00}{写PeterlitsDoc的帮助文档}{写了还蛮久的,主要是
    写到一半的时候突然就不小心被我\verb|git checkout|了,伤心。}
\item[v]{16:20}{选体育课}{向王莉老师问了一问,结果老师不在线。
    电话也没有打通。\\ 哇!老师回我了,事情解决}
\item[v]{17:18}{数学课 - 正向级数}{看完了,但是还没有跑步,还很饿,
    打算今天先回去了。}
\othe[ ]{17:40}{回家}{}
\item[v]{19:41}{出去锻炼}{说明出去不能带手机。今天锻炼得不好。}
\othe[ ]{22:48}{吃饭洗澡睡觉}{}
\othe[ ]{7:00}{起床洗漱去图书馆}{今天也起晚了,手机没有电了啦。}
\item[x]{9:30}{数学作业 - 正向级数}{奇奇怪怪,然后我只做了0.5页。}
\item[x]{10:46}{数学作业 - 正向级数}{又做了0.5页。打算待会看看中医,
    然后再做做物理。}
\item[x]{11:38}{中医学作业 - 第三章}{看了一半。}
\othe[ ]{12:14}{吃饭}{吃完饭,然后看看物理,就可以交表了。}
\item[v]{13:00}{处理停机问题}{写了一点点。停机问题、罗素悖论
    本质上的逻辑漏洞都发生在自己身上。}
\item[v]{14:25}{背单词}{快补作业!!!为什么背得这么快呀!}
\item[v]{14:30}{中医学作业 - 第三章}{看完了第三章!}
\end{pltplan}

\newpage

%%%%%%%%%%%%%%%%%%%%%%%%%%%%%%%%%%%%%%%%%%%%%%%%%%%%%%%%%%%%%%%%%%%%%

\end{document}

